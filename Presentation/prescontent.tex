\documentclass[14pt]{beamer}
\usepackage[T2A]{fontenc}
\usepackage[utf8]{inputenc}
\usepackage[english,russian]{babel}
\usepackage{amssymb,amsfonts,amsmath,mathtext}
\usepackage{cite,enumerate,float,indentfirst}

\graphicspath{{../images/}{images/}} 



% \usetheme[secheader]{Boadilla}
% \usecolortheme{seahorse}

\usetheme{Pittsburgh}
\usecolortheme{whale}

\beamertemplatenavigationsymbolsempty

\newcommand{\todo}{\alert}
%%% Основные сведения %%%
\newcommand{\thesisAuthorLastName}{Кулагин}
\newcommand{\thesisAuthorOtherNames}{Алексей Владимирович}
\newcommand{\thesisAuthorInitials}{\fixme{И.\,О.}}
\newcommand{\thesisAuthor}             % Диссертация, ФИО автора
{%
    \texorpdfstring{% \texorpdfstring takes two arguments and uses the first for (La)TeX and the second for pdf
        \thesisAuthorLastName~\thesisAuthorOtherNames% так будет отображаться на титульном листе или в тексте, где будет использоваться переменная
    }{%
        \thesisAuthorLastName, \thesisAuthorOtherNames% эта запись для свойств pdf-файла. В таком виде, если pdf будет обработан программами для сбора библиографических сведений, будет правильно представлена фамилия.
    }
}
\newcommand{\thesisAuthorShort}        % Диссертация, ФИО автора инициалами
{\thesisAuthorInitials~\thesisAuthorLastName}
%\newcommand{\thesisUdk}                % Диссертация, УДК
%{\fixme{xxx.xxx}}
\newcommand{\thesisTitle}              % Диссертация, название
{КОМПЬЮТЕРНОЕ МОДЕЛИРОВАНИЕ КВАНТОВЫХ ЭФФЕКТОВ В КОНЕЧНОМЕРНЫХ МОДЕЛЯХ}
\newcommand{\thesisSpecialtyNumber}    % Диссертация, специальность, номер
{1.2.2}
\newcommand{\thesisSpecialtyTitle}     % Диссертация, специальность, название (название взято с сайта ВАК для примера)
{\fixme{Технология обработки, хранения и~переработки злаковых, бобовых культур,
крупяных продуктов, плодоовощной продукции и~виноградарства}}
%% \newcommand{\thesisSpecialtyTwoNumber} % Диссертация, вторая специальность, номер
%% {\fixme{XX.XX.XX}}
%% \newcommand{\thesisSpecialtyTwoTitle}  % Диссертация, вторая специальность, название
%% {\fixme{Теория и~методика физического воспитания, спортивной тренировки,
%% оздоровительной и~адаптивной физической культуры}}
\newcommand{\thesisDegree}             % Диссертация, ученая степень
{кандидата физико-математических наук}
\newcommand{\thesisDegreeShort}        % Диссертация, ученая степень, краткая запись
{\fixme{канд. физ.-мат. наук}}
\newcommand{\thesisCity}               % Диссертация, город написания диссертации
{Москва}
\newcommand{\thesisYear}               % Диссертация, год написания диссертации
{\the\year}
\newcommand{\thesisOrganization}       % Диссертация, организация
{\fixme{Федеральное государственное автономное образовательное учреждение высшего
образования <<Длинное название образовательного учреждения <<АББРЕВИАТУРА>>}}
\newcommand{\thesisOrganizationShort}  % Диссертация, краткое название организации для доклада
{\fixme{НазУчДисРаб}}

\newcommand{\thesisInOrganization}     % Диссертация, организация в предложном падеже: Работа выполнена в ...
{\fixme{учреждении с~длинным длинным длинным длинным названием, в~котором
выполнялась данная диссертационная работа}}

%% \newcommand{\supervisorDead}{}           % Рисовать рамку вокруг фамилии
\newcommand{\supervisorFio}              % Научный руководитель, ФИО
{Ожигов Ю. И.}
\newcommand{\supervisorRegalia}          % Научный руководитель, регалии
{д. ф.-м. н., профессор}
\newcommand{\supervisorFioShort}         % Научный руководитель, ФИО
{\fixme{И.\,О.~Фамилия}}
\newcommand{\supervisorRegaliaShort}     % Научный руководитель, регалии
{\fixme{уч.~ст.,~уч.~зв.}}

%% \newcommand{\supervisorTwoDead}{}        % Рисовать рамку вокруг фамилии
%% \newcommand{\supervisorTwoFio}           % Второй научный руководитель, ФИО
%% {\fixme{Фамилия Имя Отчество}}
%% \newcommand{\supervisorTwoRegalia}       % Второй научный руководитель, регалии
%% {\fixme{уч. степень, уч. звание}}
%% \newcommand{\supervisorTwoFioShort}      % Второй научный руководитель, ФИО
%% {\fixme{И.\,О.~Фамилия}}
%% \newcommand{\supervisorTwoRegaliaShort}  % Второй научный руководитель, регалии
%% {\fixme{уч.~ст.,~уч.~зв.}}

\newcommand{\opponentOneFio}           % Оппонент 1, ФИО
{\fixme{Фамилия Имя Отчество}}
\newcommand{\opponentOneRegalia}       % Оппонент 1, регалии
{\fixme{доктор физико-математических наук, профессор}}
\newcommand{\opponentOneJobPlace}      % Оппонент 1, место работы
{\fixme{Не очень длинное название для места работы}}
\newcommand{\opponentOneJobPost}       % Оппонент 1, должность
{\fixme{старший научный сотрудник}}

\newcommand{\opponentTwoFio}           % Оппонент 2, ФИО
{\fixme{Фамилия Имя Отчество}}
\newcommand{\opponentTwoRegalia}       % Оппонент 2, регалии
{\fixme{кандидат физико-математических наук}}
\newcommand{\opponentTwoJobPlace}      % Оппонент 2, место работы
{\fixme{Основное место работы c длинным длинным длинным длинным названием}}
\newcommand{\opponentTwoJobPost}       % Оппонент 2, должность
{\fixme{старший научный сотрудник}}

%% \newcommand{\opponentThreeFio}         % Оппонент 3, ФИО
%% {\fixme{Фамилия Имя Отчество}}
%% \newcommand{\opponentThreeRegalia}     % Оппонент 3, регалии
%% {\fixme{кандидат физико-математических наук}}
%% \newcommand{\opponentThreeJobPlace}    % Оппонент 3, место работы
%% {\fixme{Основное место работы c длинным длинным длинным длинным названием}}
%% \newcommand{\opponentThreeJobPost}     % Оппонент 3, должность
%% {\fixme{старший научный сотрудник}}

\newcommand{\leadingOrganizationTitle} % Ведущая организация, дополнительные строки. Удалить, чтобы не отображать в автореферате
{\fixme{Федеральное государственное бюджетное образовательное учреждение высшего
профессионального образования с~длинным длинным длинным длинным названием}}

\newcommand{\defenseDate}              % Защита, дата
{\fixme{DD mmmmmmmm YYYY~г.~в~XX часов}}
\newcommand{\defenseCouncilNumber}     % Защита, номер диссертационного совета
{\fixme{Д\,123.456.78}}
\newcommand{\defenseCouncilTitle}      % Защита, учреждение диссертационного совета
{\fixme{Название учреждения}}
\newcommand{\defenseCouncilAddress}    % Защита, адрес учреждение диссертационного совета
{\fixme{Адрес}}
\newcommand{\defenseCouncilPhone}      % Телефон для справок
{\fixme{+7~(0000)~00-00-00}}

\newcommand{\defenseSecretaryFio}      % Секретарь диссертационного совета, ФИО
{\fixme{Фамилия Имя Отчество}}
\newcommand{\defenseSecretaryRegalia}  % Секретарь диссертационного совета, регалии
{\fixme{д-р~физ.-мат. наук}}            % Для сокращений есть ГОСТы, например: ГОСТ Р 7.0.12-2011 + http://base.garant.ru/179724/#block_30000

\newcommand{\synopsisLibrary}          % Автореферат, название библиотеки
{\fixme{Название библиотеки}}
\newcommand{\synopsisDate}             % Автореферат, дата рассылки
{\fixme{DD mmmmmmmm}\the\year~года}

% To avoid conflict with beamer class use \providecommand
\providecommand{\keywords}%            % Ключевые слова для метаданных PDF диссертации и автореферата
{}
      % Основные сведения

\setbeamercolor{footline}{fg=blue}
\setbeamertemplate{footline}{
  \leavevmode%
  \hbox{%
  \begin{beamercolorbox}[wd=.333333\paperwidth,ht=2.25ex,dp=1ex,center]{}%
    % И. О. Фамилия, Организация кратко
    \thesisAuthorShort, \thesisOrganizationShort
  \end{beamercolorbox}%
  \begin{beamercolorbox}[wd=.333333\paperwidth,ht=2.25ex,dp=1ex,center]{}%
    % Город, 20XX
    \thesisCity, \thesisYear
  \end{beamercolorbox}%
  \begin{beamercolorbox}[wd=.333333\paperwidth,ht=2.25ex,dp=1ex,right]{}%
  Стр. \insertframenumber{} из \inserttotalframenumber \hspace*{2ex}
  \end{beamercolorbox}}%
  \vskip0pt%
}

\newcommand{\itemi}{\item[\checkmark]}

%\title{\small{Название презентации}}
\title{\small{\thesisTitle}}
\author{\small{%
\emph{Выступающий:}~\thesisAuthorShort\\%
\emph{Руководитель:}~\supervisorRegaliaShort~\supervisorFioShort}\\%
\vspace{30pt}%
\thesisOrganization%
\vspace{20pt}%
}
\date{\small{\thesisCity, \thesisYear}}

\begin{document}

\maketitle

\begin{frame}
\frametitle{Цели и задачи}
\begin{itemize}
  \item \textbf{Предмет исследования:} 
  \item \textbf{Исследуемые характеристики:} 
  \item \textbf{Цель исследования:} 
  \item \textbf{Актуальность:} 
\end{itemize}
\end{frame}

\begin{frame}
\frametitle{Проблемы}
\begin{itemize}
  \item Проблема 1
  \item Проблема 2
  \item Проблема 3    
\end{itemize}
\end{frame}

\begin{frame}
\frametitle{План работ}
\begin{enumerate}
  \item \textbf{Задача 1}
  \begin{itemize}
    \item Подзадача 1-1
    \item Подзадача 1-2
  \end{itemize}
  \item \textbf{Задача 2}
  \begin{itemize}
    \item Подзадача 2-1
    \item Подзадача 2-2
    \item Подзадача 2-3
  \end{itemize}
  \item \textbf{Задача 3}
  \begin{itemize}
    \item Подзадача 3-1
    \item Подзадача 3-2
    \item Подзадача 3-3
  \end{itemize}
\end{enumerate}
\end{frame}

\begin{frame}
\frametitle{Список обыкновенный}
\begin{itemize}
  \item Пункт 1
  \item Пункт 2
  \item Пункт 3
\end{itemize}
\end{frame}

\begin{frame}
\frametitle{Одиночное изображение}
\begin{figure}[H]
  \center
  \includegraphics[width=0.8\linewidth]{latex}
\end{figure}
\end{frame}

\begin{frame}
\frametitle{Формулы}
$$
\left\{
  \begin{array}{rl}
    \dot x = & \sigma (y-x) \\
    \dot y = & x (r - z) - y \\
    \dot z = & xy - bz
  \end{array}
\right.
$$
\end{frame}

\begin{frame}
\frametitle{Составное изображение}
\begin{figure}[h]
  \begin{minipage}[h]{0.49\linewidth}
    \textbf{Составная \\ подпись 1}
    \center{\includegraphics[width=1\linewidth]{knuth1}}
  \end{minipage}
  \hfill
  \begin{minipage}[h]{0.49\linewidth}
    \textbf{Составная \\ подпись 2}
    \center{\includegraphics[width=1\linewidth]{knuth2}}
  \end{minipage}
\end{figure}
\end{frame}

\begin{frame}
\frametitle{Таблица}
\begin{tabular}{|l|l|}
\hline
\textbf{Заголовок 1} & \textbf{Заголовок 2} \\
\hline
Сумма & $b+a$ \\
\hline
Разность & $a-b$ \\
\hline
Произведение & $a*b$ \\
\hline
\end{tabular}
\end{frame}

\begin{frame}
\frametitle{Большой многоуровневый список}
\begin{itemize}
  \item \textbf{Пункт 1}
    \begin{itemize}
      \itemi Подпункт 1-1
      \itemi Подпункт 1-2
    \end{itemize}
  \item \textbf{Пункт 2}
    \begin{itemize}
      \itemi Подпункт 2-1
    \end{itemize}
  \item \textbf{Пункт 3}
    \begin{itemize}
      \itemi Подпункт 3-1
      \itemi Подпункт 3-2
    \end{itemize}
  \item \textbf{Пункт 4}
    \begin{itemize}
      \itemi Подпункт 4-1
    \end{itemize}
  \item \textbf{Пункт 5}
    \begin{itemize}
      \itemi Подпункт 5-1
      \itemi Подпункт 5-2
      \itemi Подпункт 5-3
    \end{itemize}
\end{itemize}
\end{frame}

\begin{frame}
\frametitle{Четыре изображения}
\begin{figure}[H]
  \center
    \includegraphics[width=0.4\linewidth]{latex}
    \includegraphics[width=0.4\linewidth]{latex}\\
    \includegraphics[width=0.4\linewidth]{latex}
    \includegraphics[width=0.4\linewidth]{latex}
\end{figure}
\end{frame}

%%%%%%%%%%%%%%%%%%%%%%%%%%%%%%
\begin{frame}
\frametitle{Перспективы развития проекта}
\begin{itemize}
  \item Перспектива 1
  \item Перспектива 2
  \item Перспектива 3
  \item Перспектива 4
  \item Перспектива 5
\end{itemize}
\end{frame}

\begin{frame}
\frametitle{Результаты работы}
\begin{itemize}
  \item Результат 1
  \item Результат 2
  \item Результат 3
  \item Результат 4
\end{itemize}
\end{frame}

\begin{frame}
\begin{center}
Спасибо за внимание!
\end{center}
\end{frame}

\end{document} 
