%%% Кодировки и шрифты %%%
\ifxetexorluatex
    % Язык по-умолчанию русский с поддержкой приятных команд пакета babel
    \ifnewpoly
        \setmainlanguage[babelshorthands=true,indentfirst=true]{russian}
    \else
        \setmainlanguage[babelshorthands=true]{russian}
    \fi
    \setotherlanguage{english}                         % Дополнительный язык = английский (в американской вариации по-умолчанию)

    % Проверка существования шрифтов. Недоступна в pdflatex
    \ifnumequal{\value{fontfamily}}{1}{
        \IfFontExistsTF{Times New Roman}{}{\setcounter{fontfamily}{0}}
    }{}
    \ifnumequal{\value{fontfamily}}{2}{
        \IfFontExistsTF{LiberationSerif}{}{\setcounter{fontfamily}{0}}
    }{}

    \ifnumequal{\value{fontfamily}}{0}{                    % Семейство шрифтов CMU. Используется как fallback
        \setmonofont{CMU Typewriter Text}                  % моноширинный шрифт
        \newfontfamily\cyrillicfonttt{CMU Typewriter Text} % моноширинный шрифт для кириллицы
        \defaultfontfeatures{Ligatures=TeX}                % стандартные лигатуры TeX, замены нескольких дефисов на тире и т. п. Настройки моноширинного шрифта должны идти до этой строки, чтобы при врезках кода программ в коде не применялись лигатуры и замены дефисов
        \setmainfont{CMU Serif}                            % Шрифт с засечками
        \newfontfamily\cyrillicfont{CMU Serif}             % Шрифт с засечками для кириллицы
        \setsansfont{CMU Sans Serif}                       % Шрифт без засечек
        \newfontfamily\cyrillicfontsf{CMU Sans Serif}      % Шрифт без засечек для кириллицы
    }

    \ifnumequal{\value{fontfamily}}{1}{                    % Семейство MS шрифтов
        \setmonofont{Courier New}                          % моноширинный шрифт
        \newfontfamily\cyrillicfonttt{Courier New}         % моноширинный шрифт для кириллицы
        \defaultfontfeatures{Ligatures=TeX}                % стандартные лигатуры TeX, замены нескольких дефисов на тире и т. п. Настройки моноширинного шрифта должны идти до этой строки, чтобы при врезках кода программ в коде не применялись лигатуры и замены дефисов
        \setmainfont{Times New Roman}                      % Шрифт с засечками
        \newfontfamily\cyrillicfont{Times New Roman}       % Шрифт с засечками для кириллицы
        \setsansfont{Arial}                                % Шрифт без засечек
        \newfontfamily\cyrillicfontsf{Arial}               % Шрифт без засечек для кириллицы
    }

    \ifnumequal{\value{fontfamily}}{2}{                    % Семейство шрифтов Liberation (https://pagure.io/liberation-fonts)
        \setmonofont{LiberationMono}[Scale=0.87] % моноширинный шрифт
        \newfontfamily\cyrillicfonttt{LiberationMono}[     % моноширинный шрифт для кириллицы
            Scale=0.87]
        \defaultfontfeatures{Ligatures=TeX}                % стандартные лигатуры TeX, замены нескольких дефисов на тире и т. п. Настройки моноширинного шрифта должны идти до этой строки, чтобы при врезках кода программ в коде не применялись лигатуры и замены дефисов
        \setmainfont{LiberationSerif}                      % Шрифт с засечками
        \newfontfamily\cyrillicfont{LiberationSerif}       % Шрифт с засечками для кириллицы
        \setsansfont{LiberationSans}                       % Шрифт без засечек
        \newfontfamily\cyrillicfontsf{LiberationSans}      % Шрифт без засечек для кириллицы
    }

\else
    \ifnumequal{\value{usealtfont}}{1}{% Используется pscyr, при наличии
        \IfFileExists{pscyr.sty}{\renewcommand{\rmdefault}{ftm}}{}
    }{}
\fi
