
\section*{Общая характеристика работы}

\newcommand{\actuality}{\underline{\textbf{\actualityTXT}}}
\newcommand{\progress}{\underline{\textbf{\progressTXT}}}
\newcommand{\aim}{\underline{{\textbf\aimTXT}}}
\newcommand{\tasks}{\underline{\textbf{\tasksTXT}}}
\newcommand{\novelty}{\underline{\textbf{\noveltyTXT}}}
\newcommand{\influence}{\underline{\textbf{\influenceTXT}}}
\newcommand{\methods}{\underline{\textbf{\methodsTXT}}}
\newcommand{\defpositions}{\underline{\textbf{\defpositionsTXT}}}
\newcommand{\reliability}{\underline{\textbf{\reliabilityTXT}}}
\newcommand{\probation}{\underline{\textbf{\probationTXT}}}
\newcommand{\contribution}{\underline{\textbf{\contributionTXT}}}
\newcommand{\publications}{\underline{\textbf{\publicationsTXT}}}
\newcommand{\object}{\underline{\textbf{\objectTXT}}}
\newcommand{\subject}{\underline{\textbf{\subjectTXT}}}

%%% Обоснование выбора названия диссертации
Молекулярная газодинамика "--- это газовая динамика, построенная на основе кинетической теории газа.
Под последней обычно понимают теорию неравновесных свойств газа.
Ключевую роль при описании газа играет отношение длины свободного пробега молекул газа \(\ell\)
к характерному размеру течения \(L\) "--- так называемое число Кнудсена \(\Kn=\ell/L\).
В континуальном пределе (\(\Kn\to0\)) обычно используют законы классической гидродинамики,
основанной на модели сплошной среды, и только в случае конечных \(\Kn\)
учитывают молекулярную структуру газа. Таким образом, в литературе можно встретить
разделение на континуальную гидрогазодинамику и динамику разрежённого газа.
Однако имеется достаточно широкий круг задач, для которых уравнения Навье"--~Стокса
некорректно описывают поведение газа даже при \(\Kn\to0\).
Поэтому в настоящем исследовании используется термин \emph{молекулярная газодинамика},
подчёркивая тот факт, что методы и представления кинетической теории газа используются
как для разрежённого газа, так и для его континуального предела.
Этот термин, по-видимому, впервые предложен в 1970 году М.\,Н.~Коганом~\autocite{Kogan1971review},
позже подхвачен в фундаментальных трудах Г.~Бёрдом~\autocite{Bird1981} и Ё.~Соне~\autocite{Sone2007}.

{\actuality}
%%% Современные проблемы и вызовы
Динамика разрежённого газа, как наука, получила своё бурное развитие с середины XX века
в связи в активным освоением космоса. Первые исследования носили в основном экспериментальный
характер. В XXI веке превалирующую роль играет компьютерное моделирование, что
говорит о зрелости теоретических представлений дисциплины.
Нельзя тоже самое сказать про строгую математическую теорию,
однако за последние три десятилетия были достигнуты значительные успехи.
Разрежённый газ описывается кинетическим уравнением Больцмана,
численное решение которого представляет собой крайне трудоёмкую и ресурсозатратную задачу.
До сих пор львиная доля исследований посвящается попыткам упрощения уравнения Больцмана,
позволяющая избежать прямого статистического моделирования (DSMC) "--- основного
на сегодняшний день метода численного анализа поведения разрежённого газа.
Однако для чисто \emph{стохастических} методов характерны значительные флуктуации,
что сильно ограничивает точность получаемых результатов.
Стремительный рост вычислительных мощностей, доступных исследователям и инженерам,
спровоцировал системное развитие широкого класса \emph{детерминистических} подходов.

%%% Про асимтотические методы
Немаловажное место занимает асимптотическая теория уравнения Больцмана,
в рамках которой слаборазрежённый газ (\(\Kn\ll1\)) может быть корректно описан с помощью
подходящей системы гидродинамических уравнений, соответствующих им граничных условий,
а также специальных поправок в кинетических слоях (кнудсеновский, ударный, начальный и др.).
Использование нескольких макроскопических переменных вместо дискретизированной функции распределения
молекулярных скоростей существенно упрощает численный анализ.

%%% Прикладные области
На сегодняшний день можно выделить несколько прикладных областей, где активно применяется
молекулярная газодинамика:
\begin{enumerate}
    \item Аэрокосмические исследования.
    Движение аппаратов в верхних слоях атмосферы соответствует задачам с конечными числами Кнудсена.
    \item Течения газа в микро- и наноэлектромеханических системах (МЭМС и НЭМС).
    Эта относительно молодая отрасль обуславливает основную волну интереса
    к изучению разрежённого газа в начале XXI века и включает широкий круг задач:
    от протекания газа через неравномерно нагретые микроканалы до процессов испарения и конденсации.
    \item Экологические проблемы.
    Конечная фаза существования атмосферных загрязнений "--- это аэрозольные частицы.
    Процесс их образования, а также изменение их дисперсного состава описываются уравнением Больцмана.
    \item Вакуумные технологии.
    Моделирование течений газа, когда число Кнудсена значительно меняется в пространственно-временных масштабах,
    представляет собой особенно трудную задачу, однако современный уровень развития вычислительных средств
    позволяет во многих случаях обходиться без дорогостоящих экспериментальных прототипов.
\end{enumerate}

Таким образом, актуальность данного исследования обусловлена следующими обстоятельствами:
\begin{itemize}
    \item активным развитием, в первую очередь, микро- и нанофлюидики, а также других прикладных областей;
    \item потребностью в высокоточных численных методах;
    \item быстрым ростом доступных вычислительных ресурсов.
\end{itemize}

{\object} "--- движение одноатомного газа различной степени разрежённости.
В исследовании одновременно изучается два {\subject}:
\begin{enumerate}
    \item Физические свойства, включая картины течений, множество сил,
    которым подвергаются обтекаемые тела, а также сопровождающие процессы теплообмена.
    \item Методы численного и асимптотического анализа.
\end{enumerate}

{\progress}

%%% пространственно-однородная задача Коши
Методы численного анализа и асимптотическая теория уравнения Больцмана тесно связаны
с математической теорией задачи Коши.
Пионерские работы, посвящённые \(L^1\)-теории в пространственно однородной постановке,
принадлежат Т.~Карлеману~\autocite{Carleman1933}, Д.~Моргенштерну~\autocite{Morgenstern1954},
А.\,Я.~Повзнеру~\autocite{Povzner1962} и Л.~Аркерюду~\autocite{Arkeryd1972}.
Теория Фурье"=преобразования уравнения Больцмана Обширная программа анализа максвелловского газа на основе
была реализована А.\,В.~Бобылевым~\autocite{Bobylev1984}.
Математические инструменты для анализа сингулярных больцмановских ядер,
характерных для дальнодействующих потенциалов, появились на рубеже веков как результат труда
П.-Л.~Лионса, Р.~Александре, Л.~Девиллета, Б.~Веннберга и С.~Виллани~\autocite{Lions1989, Alexandre2000}.

%%% пространственно-неоднородная задача Коши
Для полного уравнения Больцмана первая и единственная на сегодняшний день \(L^1\)-теория
построена на основе ренормализованных решений Р.~ди-Перна и П.-Л.~Лионсом~\autocite{Lions1989}.
Почти экспоненциальная сходимость к равновесию \(\OO{t^{-\infty}}\) с явными оценками
для любого достаточно гладкого решения была доказана Л.~Девиллетом и С.~Виллани~\autocite{Villani2005}.
Оба этих результата отмечены Филдсовскими премиями 1994 и 2010 годов.
Строгая линеаризованная теория была построена Х.~Грэдом~\autocite{Grad1963b}.
Вопрос асимптотической устойчивости глобального максвелловского распределения
получил первый положительный ответ для короткодействующих жёстких потенциалов ещё в 1974 году~\autocite{Ukai1974},
но был окончательно закрыт в недавних работах Ф.~Грессмана, Р.~Стрейна~\autocite{Strain2011} и независимо
Р.~Александре, Ё.~Моримото, С.~Юкая, Ч.-Ц.~Сюя, Т.~Янга~\autocite{Alexandre2012soft}.
Т.-П.~Лю и Ш.-С.~Ю внесли важный вклад в понимании больцмановской динамики
на основе анализа функции Грина линеаризованного уравнения Больцмана~\autocite{Liu2004green, Liu2006}.

%%% формальная асимптотическая теория
Формальная асимптотическая теория уравнения Больцмана была заложена с трудах Д.~Гильберта~\autocite{Hilbert1912},
С.~Чепмена~\autocite{Chapman1916}, Д.~Энскога~\autocite{Enskog1917},
позже развита Д.~Барнеттом~\autocite{Burnett1935}, Х.~Грэдом~\autocite{Grad1963a} и Ё.~Соне~\autocite{Sone2002}.
Решение уравнения Больцмана для слаборазрежённого газа допускает отделение гидродинамической части
от существенно неравновесных пространственно"=временн\'{ы}х \emph{кинетических} слоёв.
В зависимости от способа масштабирования различные системы гидродинамических уравнений могут быть получены.
В частности, для медленных неизотермических течений справедливы
\emph{уравнения Когана"--~Галкина"--~Фридлендера}~\autocite{Kogan1976} (\emph{КГФ}),
содержащие некоторые барнеттовские члены.

%%% строгая асимптотическая теория
Строгая асимптотическая теория тесно связана с развитием математической теории самих гидродинамических уравнений.
Существование слабых решений уравнений Навье"--~Стокса для несжимаемой жидкости было показано Ж.~Лер\'{е}~\autocite{Leray1934}.
Сходимость к ним ренормализованных решений ди-Перна"--~Лионса установлена в трудах
К.~Бардоса, Ф.~Гольса, Д.~Левермора~\autocite{Bardos1993}, П.-Л.~Лионса, Н.~Масмоуди~\autocite{Masmoudi2001} и Л.~Сен-Ремон~\autocite{Golse2004}.
Строгая асимптотическая теория в пределе сжимаемой жидкости далека от своей зрелости.
Частичные результаты о сходимости к гладким решениям уравнений Эйлера принадлежат
Т.~Нишиде~\autocite{Nishida1978} и Р.~Кафлишу~\autocite{Caflisch1980limit}.
В присутствии процессов высокой частоты (сравнимой с частотой столкновений молекул) больцмановская динамика
качественно отличается от классической гидродинамики на основе линейных законов Ньютона и Фурье.
На основе множества работ И.\,В.~Карлина, А.\,Н.~Горбаня, М.~Слемрода совместно с другими авторами,
можно сделать вывод, что в пределе малых чисел Кнудсена и конечных числах Маха корректные уравнения
гидродинамического типа должны демонстрировать константную диссипацию высокочастотных мод
и существенно нелокальный характер~\autocite{Gorban2014}.

%%% кинетические слои
\emph{Пограничные} кинетические слои моделируются краевыми задачами в полупространстве~\autocite{Grad1969}.
Соответствующие им теоремы существования и единственности решения линеаризованного уравнения Больцмана для газа твёрдых сфер
были доказаны Н.\,Б.~Масловой~\autocite{Maslova1982} и независимо К.~Бардосом, Р.~Кафлишем, Б.~Николаенко~\autocite{Bardos1986}.
Пограничный слой с конденсацией и испарением изучен К.~Черчиньяни~\autocite{Cercignani1986} и учениками К.~Бардоса~\autocite{Coron1988}.
Нелинейная теория заложена в трудах Л.~Аркерюда, А.~Нури~\autocite{Arkeryd2000},
С.~Юкая, Т.~Янга, Ш.-С.~Ю~\autocite{Ukai2003} и Ф.~Гольса~\autocite{Golse2008}.
Большой цикл работ Киотской группы (Ё.~Соне, К.~Аоки, Ш.~Таката, Т.~Овада и др.)
посвящён высокоточному численному анализу кнудсеновского слоя первого~\autocite{Ohwada1989creep, Ohwada1989jump}
и второго порядка~\autocite{Ohwada1992, Takata2015second, Takata2015curvature}
для диффузного отражения и газа твёрдых сфер.
Кинетическая теория ударных волн развита только для малых амплитуд.
Для жёстких короткодействующих потенциалов ударный профиль впервые был построен Р.~Кафлишем и Б.~Николаенко~\autocite{Caflisch1982}.
Cтабильность, позитивность~\autocite{Liu2004} и монотонность~\autocite{Liu2013} была показана Т.-П.~Лю и Ш.-С.~Ю.

%%% численные методы
В трудах Киотской группы реализована масштабная программа численного анализа классических задач
динамики разрежённого газа на основе линеаризованного уравнения Больцмана для газа твёрдых сфер.
Широкий круг нелинейных задач изучен с высокой точностью для упрощённого столкновительного оператора,
предложенного П.~Бхатнагаром, Е.~Гроссом, М.~Круком~\autocite{Krook1954} и независимо П.~Веландером~\autocite{Welander1954}.
Огромное множество исследований посвящено непосредственному решению уравнения Больцмана.
Среди них можно выделить три магистральных направления в зависимости от способа аппроксимации
функции распределения скоростей.
\begin{enumerate}
    \item Стохастические методы основаны на случайных процессах марковского типа.
    \item Методы дискретных скоростей подразумевают конечное множество доступных молекулярных скоростей.
    \item Проекционные (spectral) методы используют разложение по базисным функциям.
\end{enumerate}

%%% стохастические методы
В настоящее время для численного решения существенно нелинейных задач наиболее распространён
метод \emph{прямого статистического моделирования} (DSMC),
впервые описанный Г.~Бёрдом~\autocite{Bird1963}.
Современные его реализации используют улучшенные алгоритмы выбора сталкивающихся частиц,
предложенные позже самим Г.~Бёрдом~\autocite{Bird1989}, М.\,С.~Ивановым и С.\,В.~Рогазинским~\autocite{Ivanov1988}.
Сходимость метода доказана В.~Вагнером~\autocite{Wagner1992}.
Альтернативные \emph{стохастические модели} были в своё время разработаны
В.\,Е.~Яницким и О.\,М.~Белоцерковским~\autocite{Yanitskij1975},
К.~Нанбу~\autocite{Nanbu1980} и Х.~Бабовским~\autocite{Babovsky1986}.

%%% методы дискретных скоростей
Фиксированная сетка в пространстве молекулярных скоростей для решения уравнения Больцмана
была впервые использована Дж.~Хэвилендом~\autocite{Haviland1965, Haviland1970},
а также А.~Нордсиком и Б.~Хиксом~\autocite{Nordsieck1966, Nordsieck1970}.
Для вычисления интеграла столкновения они использовали кубатуры Монте"=Карло с последующей
консервативной коррекцией функции распределения.
В дальнейшем \emph{метод дискретных скоростей} развивался С.~Йеном~\autocite{Yen1984},
В.\,В.~Аристовым и Ф.\,Г.~Черемисиным~\autocite{Tcheremissine1980}.
Чисто детерминистические модели дискретных скоростей берут начало с работ
Т.~Карлемана~\autocite{Carleman1957} и Дж.~Бродуэлла~\autocite{Broadwell1964shock, Broadwell1964shear}.
Методы построения таких моделей изучались С.\,К.~Годуновым и У.\,М.~Султангазиным~\autocite{Sultangazin1971},
Р.~Гатиньоль~\autocite{Gatignol1975} и А.~Кабанном~\autocite{Cabannes1980},
А.\,В.~Бобылевым и К.~Черчиньяни~\autocite{Bobylev1999dvm}.
Сходимость дискретного газа к решениям Ди-Перна"--~Лионса доказана С.~Мишлером~\autocite{Mischler1997}.

%%% современные методы дискретных скоростей
А.~Пальцевский, Ж.~Шнайдер и А.\,В.~Бобылев показали, что классические модели дискретных скоростей,
несмотря на присущие им \emph{консервативность} на микроскопическом уровне и \emph{энтропийность}
(выполнение \(\mathcal{H}\)-теоремы), обладают дробным порядком сходимости~\autocite{Palczewski1997}.
В.~Панфёров и А.~Гейнц показали, как Карлеманова замена переменных
позволяет улучшить сходимость, но лишь вплоть до первого порядка~\autocite{Panferov2002}.
Пространственное размазывание (mollification) столкновительной сферы позволяет естественным образом
решить проблему консервативной дискретизации, избегая решения целочисленных уравнений.
К.~Бюе, С.~Кордье и П.~Дегон продемонстрировали, как обеспечить консервативность на макроскопическом уровне
(для столкновительного оператора целиком)~\autocite{Buet1998};
Х.~Бабовски построил простейшую схему с консервативностью на мезоскопическом уровне
(для всей столкновительной сферы)~\autocite{Babovsky1998}, его подход позже развил Д.~Гёрш~\autocite{Goersch2002};
наконец, Ф.\,Г.~Черемисин предложил новый класс методов, сохраняющих консервативность на микроскопическом уровне
(для отдельной столкновительной пары)~\autocite{Tcheremissine1997}.
Вычислительная сложность методов дискретных скоростей может быть значительно снижена
с помощью теоретикочисловых (quasi-Monte Carlo) методов кратного интегрирования,
впервые разработанных Н.\,М.~Коробовым~\autocite{Korobov1959}.

%%% проекционные методы
Дополнительное априорное знание о функции распределения в отдельном классе задач может служить
основой для построения более эффективных, но менее универсальных численных методов.
Проекционные методы вычисления интеграла столкновений берут своё начало с классического
13-моментного метода Х.~Грэда~\autocite{Grad1949}, который представляет собой метод Галёркина
с многочленами Эрмита в качестве базиса и локальным Максвеллианом в качестве весовой функции.
И.\,А.~Эндер и А.\,Я.~Эндер использовали полиномы Сонина для приближённого решения изотропного уравнения Больцмана~\autocite{Ender1970}.
А.\,В.~Бобылев показал, что столкновительный оператор в Фурье-базисе принимает особенно простой вид
для максвелловских молекул~\autocite{Bobylev1975}.
С вычислительной точки зрения такой базис привлекален благодаря алгоритмам быстрого преобразования Фурье.
Первые численные результаты на основе метода Фурье"--~Галёркина принадлежат
Ю.\,Н.~Григорьеву и А.\,Н.~Михалицыну~\autocite{Grigoriev1983};
для произвольного потенциала подход был обобщён Л.~Парески и Дж.~Руссо~\autocite{Pareschi2000method},
а также Н.\,Х.~Ибрагимовым и С.\,В.~Рязановым~\autocite{Ibragimov2002}.
К.~Муо и Л.~Парески показали, как можно понизить вычислительную сложность метода Фурье"--~Галёркина
на основе представления Карлемана и декомпозиции столкновительного ядра~\autocite{Pareschi2006}.
Проекционный метод имеет \emph{экспоненциальную сходимость}, однако не сохраняет положительность, импульс и энергию.
Полная консервативность может быть обеспечена различными процедурами коррекции решения,
а Ф.~Фильбе и К.~Муо удалось доказать асимптотическую стабильность возмущённого столкновительного оператора,
не сохраняющего положительность~\autocite{Filbet2011}.

%%% метод Черемисина = проекционный метод дискретных скоростей
Наконец, метод дискретных скоростей может быть интерпретирован как проекционный метод в пространстве дельта-функций.
Микроскопическая консервативность, полученная Ф.\,Г.~Черемисиным, достигается методом Петрова"--~Галёркина,
в котором столкновительные инварианты образуют ортогональную оболочку.
Далее, метод Черемисина будем называть \emph{проекционным методом дискретных скоростей}.

В свете упомянутых выдающихся достижений кинетической теории, настоящее исследование лишь скромно
касается некоторых проблем молекулярной газовой динамики. В частности, выделены две основные {\aim}:
\begin{enumerate}
    \item Развитие проекционного метода дискретных скоростей для неравномерных сеток,
    его верификация в широком диапазоне неравновесности.
    \item Численный анализ некоторых классических течений разрежённого газа на основе
    как уравнения Больцмана, так и соответствующих уравнений гидродинамического типа.
    Оценка области применимости последних при различных граничных условиях.
\end{enumerate}
Для достижения поставленных целей поставлены следующие {\tasks}:
\begin{enumerate}
    \item Оценка точности проекционного метода дискретных скоростей на различных численных примерах
    и выделение круга задач, где использование неравномерных сеток необходимо и оправдано.\label{itm:task_first1}
    \item Изучение и сравнительный анализ многоточечных проекционных шаблонов,
    необходимых для консервативного вычисления интеграла столкновений на неравномерных сетках.
    \item Построение асимптотического решения второго порядка
    для пограничного слоя Прандтля для газа твёрдых сфер. \label{itm:task_first2}
    \item Сравнительный анализ численных решений задачи Куэтта в широком диапазоне параметров,
    получаемых с помощью проекционного метода и других общепризнанных методов.
    \item Исследование сходимости численного решения уравнения Больцмана к асимптотическому
    для широкого класса течений между параллельными пластинами.
    \item Исследование различных подходов к постановке граничных условий для уравнений КГФ,
    сравнительный анализ с решением уравнения Больцмана.\label{itm:task_last1}
    \item Разработка комплекса программ для решения уравнений КГФ в произвольной геометрии.
    \item Параметрический анализ течений между некоаксиальными и эллиптическими цилиндрами
    в континуальном пределе.\label{itm:task_last2}
\end{enumerate}
Задачи \ref{itm:task_first1}--\ref{itm:task_last1} позволяют достичь первой цели,
задачи \ref{itm:task_first2}--\ref{itm:task_last2} раскрывают содержание второй цели.

%%% Новые методики и идеи, но не новые результаты!
{\novelty}
\begin{enumerate}
    \item Проекционный метод дискретных скоростей применяется для сильно неравномерных сеток в пространстве скоростей. % Rogozin2016
    \item Уравнения КГФ решаются с граничными условиями, содержащими члены отличные от теплового скольжения. % Rogozin2017
\end{enumerate}

{\influence}
\begin{enumerate}
    \item Проекционный метод дискретных скоростей на неравномерных сетках может быть использован
    как инструмент для численного анализа течений разрежённого газа с повышенной точностью.
    \item Результаты анализа нелинейной задачи Куэтта могут служить эталоном
    для верификации других численных методов.
    \item Разработанный солвер \verb+snitSimpleFoam+ предоставляет широкие возможности для моделирования медленных
    неизотермических течений слаборазрежённого газа в произвольной геометрии
    как в академических целях, так и для инженерных приложений.
\end{enumerate}

{\methods}
В качестве математической модели неравновесного газа используется кинетическая теория,
высокий уровень развития которой позволяет настоящему исследованию обходиться без эмпирической базы.
Методологическая база включает специальные математические и вычислительные методы:
\begin{itemize}
    \item асимптотические методы нелинейной теории возмущения;
    \item численные методы интегрирования систем дифференциальных уравнений в частных производных,
    специальные численные методы вычислительной гидродинамики;
    \item численные методы многомерного интегрирования;
    \item квадратурные методы решения интегральных уравнений;
    \item проекционные методы решения операторных уравнений;
    \item вариационное исчисление.
\end{itemize}
Численные результаты получены с использованием широкого спектра современных компьютерных технологий и программных комплексов, включая
\begin{itemize}
    \item системы компьютерной алгебры (SymPy~\autocite{sympy}),
    \item генерацию расчётных сеток (gmsh~\autocite{gmsh}),
    \item организацию параллельных вычислений (MPI~\autocite{mpi}),
    \item инструментарий вычислительной гидродинамики (OpenFOAM~\autocite{openfoam}),
    \item визуализацию полей (matplotlib~\autocite{matplotlib}).
\end{itemize}

%%% Результаты исследования
В соответствии с результатами решения поставленных задач выдвигаются {\defpositions}
\begin{enumerate}
    \item % Rogozin2011
    На численных примерах показано, что проекционный метод дискретных скоростей на равномерной сетке адекватно
    и эффективно описывает поведение разреженного газа умеренной степени неравновесности~\cite{Rogozin2011}.
    Использование неравномерной сетки оправдано а) для детального разрешения плоских кинетических слоёв,
    б) при широком температурном диапазоне.
    \item % Rogozin2016
    Для многоточечных проекционных шаблонов выявлены критерии,
    минимизирующие требования к мощности множества кубатурных точек~\cite{Rogozin2016}.
    Построены оптимальные пяти- и семиточечный шаблоны.
    \item % Rogozin2016
    С точностью 8--10 знаков вычислены неизвестные ранее транспортные коэффициенты
    (для газа твёрдых сфер), необходимые для вычисления тензора напряжений и вектора потока тепла
    при асимптотическом анализе пограничного слоя Прандтля второго порядка~\cite{Rogozin2016}.
    \item % Rogozin2016
    С помощью проекционного метода получено решение плоской задачи Куэтта,
    совпадающее с результатами прямого статистического моделирования (DSMC)
    вплоть до гиперзвуковых скоростей для широкого диапазона чисел Кнудсена~\cite{Rogozin2016}.
    \item % Rogozin2016+2017
    Проекционный метод на неравномерных сетках "--- надёжный инструмент для высокоточного анализа
    \emph{нелинейных} плоских кинетических слоёв. В частности, продемострировано отклонение от асимптотического
    решения не более \(10^{-4}\) для нелинейных течений между параллельными пластинами с температурой,
    распределённой а) константно~\cite{Rogozin2016}, б) синусоидально~\cite{Rogozin2017}.
    \item % Rogozin2017
    На численных примерах показано, что использование совместимых граничных условий
    первого и второго порядка для уравнений КГФ улучшает точность асимптотического решения
    в сравнении с решением уравнения Больцмана, полученным проекционным методом~\cite{Rogozin2017}.
    Исследованы, в том числе, граничные условия, учитывающие кривизну граничной поверхности.
    \item % Rogozin2014
    В рамках платформы OpenFOAM разработан солвер уравнений КГФ
    на основе метода конечных объёмов и модифицированного алгоритма
    решения уравнений Навье"--~Стокса для несжимающей жидкости (SIMPLE)~\cite{Rogozin2014}.
    \item % Rogozin2014 + unpublished
    На основе численного параметрического анализа некоторых \emph{нелинейных} течений газа
    между равномерно нагретыми телами в континуальном пределе была продемонстрирована
    электростатическая аналогия: обтекаемые тела притягиваются подобно электрически заряженным телам,
    при этом сила \(F\sim \Kn^2(T_2^s-T_1^s)(T_2^{1+s}-T_1^{1+s})\),
    когда вязкость и теплопроводность газа \(\mu, \lambda \sim T^s\)~\cite{Rogozin2014}.
\end{enumerate}

{\reliability} полученных результатов обеспечивается следующими обстоятельствами:
\begin{enumerate}
    \item Кинетическое уравнение Больцмана выводится из первых принципов и
    содержит минимальное количество дополнительных допущений.
    В настоящем исследовании в качестве потенциала межмолекулярного взаимодействия
    повсеместно используется модель твёрдых сфер,
    которая достаточно адекватно отражает реальные кинетические процессы в широком диапазоне неравновесности.
    В качестве модели взаимодействия газа с поверхностью используется полное диффузное отражение.
    \item Проводится систематический сравнительный анализ результатов,
    полученных с помощью проекционного метода, прямое статистического моделирования
    и асимптотическиго анализа уравнения Больцмана.
    \item Проводится анализ сходимости численных методов на основе
    множества решений на разностных сетках различной мелкости.
    \item Верификация используемых солверов и систем обработки данных
    выполнена на тестовых задачах, решение которых с высокой точностью представлено в литературе.
    Результаты находятся в полном соответствии с результатами, полученными другими авторами.
\end{enumerate}

{\probation} Результаты диссертации докладывались лично соискателем на
\begin{itemize}
    \item 54-й научной конференции МФТИ (Долгопрудный, 2011),
    \item IX международной конференции по неравновесным процессам в соплах и струях (Алушта, 2012),
    \item семинаре сектора кинетической теории отдела механики ВЦ ФИЦ ИУ РАН (Москва, 2016).
\end{itemize}

{\contribution} соискателя в работах с соавторами заключается в следующем:
разработка алгоритмов, программная реализация, проведение вычислительных экспериментов, анализ результатов.

\ifnumequal{\value{bibliosel}}{0}{% Встроенная реализация с загрузкой файла через движок bibtex8
    \publications\ Основные результаты по теме диссертации изложены в XX печатных изданиях,
    X из которых изданы в журналах, рекомендованных ВАК,
    X "--- в тезисах докладов.%
}{% Реализация пакетом biblatex через движок biber
%Сделана отдельная секция, чтобы не отображались в списке цитированных материалов
    \begin{refsection}[vak,papers,conf]% Подсчет и нумерация авторских работ. Засчитываются только те, которые были прописаны внутри \nocite{}.
        %Чтобы сменить порядок разделов в сгрупированном списке литературы необходимо перетасовать следующие три строчки, а также команды в разделе \newcommand*{\insertbiblioauthorgrouped} в файле biblio/biblatex.tex
        \printbibliography[heading=countauthorvak, env=countauthorvak, keyword=biblioauthorvak, section=1]%
        \printbibliography[heading=countauthorconf, env=countauthorconf, keyword=biblioauthorconf, section=1]%
        \printbibliography[heading=countauthornotvak, env=countauthornotvak, keyword=biblioauthornotvak, section=1]%
        \printbibliography[heading=countauthor, env=countauthor, keyword=biblioauthor, section=1]%
        \nocite{Rogozin2010, Rogozin2011, Rogozin2014, Rogozin2016, Rogozin2017}
        \publications\ Основные результаты по теме диссертации изложены в \arabic{citeauthorvak} печатных изданиях,
        рекомендованных ВАК.
    \end{refsection}
    \begin{refsection}[vak,papers,conf]%Блок, позволяющий отобрать из всех работ автора наиболее значимые, и только их вывести в автореферате, но считать в блоке выше общее число работ
        \printbibliography[heading=countauthorvak, env=countauthorvak, keyword=biblioauthorvak, section=2]%
        \printbibliography[heading=countauthornotvak, env=countauthornotvak, keyword=biblioauthornotvak, section=2]%
        \printbibliography[heading=countauthorconf, env=countauthorconf, keyword=biblioauthorconf, section=2]%
        \printbibliography[heading=countauthor, env=countauthor, keyword=biblioauthor, section=2]%
        \nocite{Rogozin2010, Rogozin2011, Rogozin2014}
    \end{refsection}
}
 % Характеристика работы по структуре во введении и в автореферате не отличается (ГОСТ Р 7.0.11, пункты 5.3.1 и 9.2.1), потому её загружаем из одного и того же внешнего файла, предварительно задав форму выделения некоторым параметрам

%Диссертационная работа была выполнена при поддержке грантов ...

%\underline{\textbf{Объем и структура работы.}} Диссертация состоит из~введения, четырех глав, заключения и~приложения. Полный объем диссертации \textbf{ХХХ}~страниц текста с~\textbf{ХХ}~рисунками и~5~таблицами. Список литературы содержит \textbf{ХХX}~наименование.

%\newpage
\section*{Содержание работы}

Во \underline{\textbf{введении}} обосновывается актуальность избранной темы, характеризуется степень ее разработанности,
определяются цели и задачи диссертационного исследования, осуществляется выбор предмета и объекта исследования,
раскрывается научная новизна, определяются методологические основания исследования,
теоретическая и практическая значимость полученных автором результатов,
формулируются положения, выносимые на защиту.

\underline{\textbf{Первая глава}} содержит математические основы кинетического описания разрежённого газа
на основе уравнения Больцмана, его линеаризации, а также широко используемых модельных уравнений.
Отдельное внимание уделено современной нелинейной асимптотической теории для слаборазрежённого газа,
где решение стационарного кинетического уравнения возможно представить в виде суммы
гидродинамического решения и поправки кинетического пограничного слоя (слоя Кнудсена).
Рассматриваются как медленные неизотермические течения с конечными числами Рейнольдса,
так и задачи с возникновением вязкого пограничного слоя (слоя Прандтля) при конечных числах Маха.
Обосновывается методика постановки третьей краевой задачи для решения
соответствующих уравнений гидродинамического типа.

\underline{\textbf{Вторая глава}} посвящена численным методам решения как уравнений гидродинамического типа,
так и непосредственно уравнения Больцмана.
Основное внимание уделяется построению консервативных разностных схем.
Для этой цели при решении систем дифференциальных уравнений в частных производных
используется метод конечных объёмов.
Для решения уравнений КГФ используется модификация алгоритма SIMPLE "--- стандартного метода решения
стационарных уравнений Навье"--~Стокса.
Излагается сравнительный обзор существующих методов решения уравнения Больцмана.
Избранный в работе подход к решению уравнения Больцмана основывается на работах Черемисина Ф.~Г.,
обобщённых на случай неравномерных сеток в пространстве скоростей.
Подробно излагается методика построения численной схемы, позволяющей достигнуть
второго порядка точности во всём семимерном фазовом пространстве:
по времени, по физическим и скоростным координатам.
Для этого используются симметричная схема расщепления, TVD-схемы для аппроксимации
бесстолкновительного уравнения Больцмана и проекционный метод дискретных скоростей
для вычисления интеграла столкновений.
Кроме стандартного двухточечного шаблона проецирования для равномерных сеток,
рассматриваются пятиточечная и семиточечная схемы.

\underline{\textbf{Третья глава}} состоит из нескольких классических задач молекулярной газовой динамики,
изучаемых с высокой точностью с помощью изложенных выше методов.
Сначала рассматривается плоское течение Куэтта для широкого диапазона чисел Кнудсена (0.01 --- 100)
и чисел Маха (0.1 -- 5). Анализируются полученные профили макроскопических величин в зависимости от
этих параметров. Верификация результатов проводится по известному численному решению линейной задачи Куэтта,
а также с помощью статистического моделирования (метод DSMC).
Кроме того, детально рассматриваются срезы функции распределения скоростей, её анизотропичность и разрывы на границе.
Задача Соне"--~Бобылева также детально изучается как в помощью асимптотического решения,
так и численного решения уравнения Больцмана. Демонстрируются ненавье"--~стоксовские эффекты,
а также сходимость кинетического решения именно к решению уравнений КГФ,
а не Навье"--~Стокса в континуальном пределе (эффект призрака).
Показывается, что решение третьей краевой задачи позволяет получить приближённое решение
для малых чисел Кнудсена, которое с хорошей точностью отражает прямое решение уравнения Больцмана.
Наконец, рассматриваются некоторые задачи термострессовой конвекции, а именно
течение газа между некоаксиальными цилиндрами и коаксиальными эллиптическими цилиндрами.
В континуальном пределе проводится параметрический анализ этих задач для различных соотношений температур,
расстоянию между осями некоаксиальных цилиндров, углу поворота между главными осями эллипсов.
Детально изучается силы, действующие на неподвижные равномерно нагретые тела со стороны газа,
имеющие второй порядок малости по числу Кнудсена.
Численное решение уравнения Больцмана свидетельствует о том, что уже для \(\Kn\gtrapprox0.01\)
тепловое скольжение существенно превалирует над термострессовой конвекцией.

В \underline{\textbf{заключении}} приведены основные результаты работы, которые заключаются в следующем:
%% Согласно ГОСТ Р 7.0.11-2011:
%% 5.3.3 В заключении диссертации излагают итоги выполненного исследования, рекомендации, перспективы дальнейшей разработки темы.
%% 9.2.3 В заключении автореферата диссертации излагают итоги данного исследования, рекомендации и перспективы дальнейшей разработки темы.
\begin{enumerate}

\item Все полученные результаты могут быть обобщены для произвольного потенциала
практически без увеличения вычислительных затрат.

\item в расширении области применимости асимтотического решения за счёт
особой методики постановки граничных условий;

\item в комплексной верификации проекционного метода дискретных скоростей
для широкого круга практических задач;

\item С точки зрения точности результатов и вычислительных затрат асимптотическое решение
является оптимальным выбором для моделирования течений в области \emph{малых чисел Кнудсена}.
Однако такое решение доступно только для узкого класса краевых задач с гладкой границей,
поэтому в большей степени оно должно служить для верификации численных методов общего назначения.
Проекционный метод на равномерной скоростной сетке "--- один из таких.
Полученные с его помощью результаты демонстрируют хорошую сходимость к асимптотическому решению,
однако требуют большого числа итераций для достижения стационарного состояния.
\item Для \emph{чисел Кнудсена близких к единице} проекционный метод позволяет получить
более точное решение по сравнению с методами прямого статистического моделирования,
для которых характерны значительные флуктуации функции распределения.
Особенно это касается малых чисел Маха, где статистические флуктуации не позволяют
получить адекватную картину течений.
\item Эффективность проекционного метода, как и любого метода дискретных скоростей,
на равномерной сетке в скоростном пространстве снижается для \emph{больших чисел Кнудсена}
ввиду значительных градиентов функции распределения во всей области, занимаемой газом.
Использование в этом случае неравномерной скоростной сетки даёт возможность
аппроксимировать функцию распределения с высокой точностью. Однако поскольку методы
дискретных скоростей подразумевают постоянство скоростной сетки во всём физическом
пространстве, то такой подход не может быть применён непосредственно к задачам
с произвольной геометрией и требует дальнейшего существенного развития.
\item Для \emph{медленных неизотермических течений} численное решение уравнения Больцмана
сходится к решению уравнений КГФ, получаемых в ходе асимптотического анализа.
При этом смешивание граничных условий разных порядков позволяет получить
температурное поле не только в континуальном пределе, но и для малых чисел Кнудсена.
В последнем случае оно является приближённым, но не асимптотическим решением
следующего порядка.

\item Проекционный метод, как и любой метод дискретных скоростей, сталкивается с проблемой
эффективной дискретизации. Области пространства скоростей,
где наблюдаются значительные перепады функции распределения, представляют известные трудности
для достижения высокой точности аппроксимации.

\item По сравнению с другими методами, проекционный метод на неравномерных сетках
позволяет достичь повышенной точности численного анализа плоского приграничного слоя
от линейных и вплоть до гиперзвуковых течений для широкого диапазона чисел Кнудсена.

\item Использование всех совместимых с уравнениями КГФ граничных условий первого и второго порядка
позволяет существенно улучшить точность асимптотического решения

\end{enumerate}


%\newpage

\ifdefmacro{\microtypesetup}{\microtypesetup{protrusion=false}}{} % не рекомендуется применять пакет микротипографики к автоматически генерируемому списку литературы
\ifnumequal{\value{bibliosel}}{0}{% Встроенная реализация с загрузкой файла через движок bibtex8
  \renewcommand{\bibname}{\large \authorbibtitle}
  \nocite{*}
  \insertbiblioauthor           % Подключаем Bib-базы
  %\insertbiblioother   % !!! bibtex не умеет работать с несколькими библиографиями !!!
}{% Реализация пакетом biblatex через движок biber
  \ifnumgreater{\value{usefootcite}}{0}{
  \insertbiblioauthorcited      % Вывод процитированных в автореферате работ автора
  }{
  \insertbiblioauthor           % Вывод всех работ автора
%  \insertbiblioauthorgrouped    % Вывод всех работ автора, сгруппированных по источникам
%  \insertbiblioauthorimportant  % Вывод наиболее значимых работ автора (определяется в файле characteristic во второй section)
  \insertbiblioother            % Вывод списка литературы, на которую ссылались в тексте автореферата
  }
}
\ifdefmacro{\microtypesetup}{\microtypesetup{protrusion=true}}{}

