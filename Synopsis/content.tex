
\section*{Общая характеристика работы}

\newcommand{\actuality}{\underline{\textbf{\actualityTXT}}}
\newcommand{\progress}{\underline{\textbf{\progressTXT}}}
\newcommand{\aim}{\underline{{\textbf\aimTXT}}}
\newcommand{\tasks}{\underline{\textbf{\tasksTXT}}}
\newcommand{\novelty}{\underline{\textbf{\noveltyTXT}}}
\newcommand{\influence}{\underline{\textbf{\influenceTXT}}}
\newcommand{\methods}{\underline{\textbf{\methodsTXT}}}
\newcommand{\defpositions}{\underline{\textbf{\defpositionsTXT}}}
\newcommand{\reliability}{\underline{\textbf{\reliabilityTXT}}}
\newcommand{\probation}{\underline{\textbf{\probationTXT}}}
\newcommand{\contribution}{\underline{\textbf{\contributionTXT}}}
\newcommand{\publications}{\underline{\textbf{\publicationsTXT}}}
\newcommand{\object}{\underline{\textbf{\objectTXT}}}
\newcommand{\subject}{\underline{\textbf{\subjectTXT}}}

\ifnumequal{\value{bibliosel}}{0}
{%%% Встроенная реализация с загрузкой файла через движок bibtex8. (При желании, внутри можно использовать обычные ссылки, наподобие `\cite{vakbib1,vakbib2}`).
    {\publications} Основные результаты по теме диссертации изложены
    в~XX~печатных изданиях,
    X из которых изданы в журналах, рекомендованных ВАК,
    X "--- в тезисах докладов.
}%
{%%% Реализация пакетом biblatex через движок biber
    \begin{refsection}[bl-author, bl-registered]
        % Это refsection=1.
        % Процитированные здесь работы:
        %  * подсчитываются, для автоматического составления фразы "Основные результаты ..."
        %  * попадают в авторскую библиографию, при usefootcite==0 и стиле `\insertbiblioauthor` или `\insertbiblioauthorgrouped`
        %  * нумеруются там в зависимости от порядка команд `\printbibliography` в этом разделе.
        %  * при использовании `\insertbiblioauthorgrouped`, порядок команд `\printbibliography` в нём должен быть тем же (см. biblio/biblatex.tex)
        %
        % Невидимый библиографический список для подсчёта количества публикаций:
        \printbibliography[heading=nobibheading, section=1, env=countauthorvak,          keyword=biblioauthorvak]%
        \printbibliography[heading=nobibheading, section=1, env=countauthorwos,          keyword=biblioauthorwos]%
        \printbibliography[heading=nobibheading, section=1, env=countauthorscopus,       keyword=biblioauthorscopus]%
        \printbibliography[heading=nobibheading, section=1, env=countauthorconf,         keyword=biblioauthorconf]%
        \printbibliography[heading=nobibheading, section=1, env=countauthorother,        keyword=biblioauthorother]%
        \printbibliography[heading=nobibheading, section=1, env=countregistered,         keyword=biblioregistered]%
        \printbibliography[heading=nobibheading, section=1, env=countauthorpatent,       keyword=biblioauthorpatent]%
        \printbibliography[heading=nobibheading, section=1, env=countauthorprogram,      keyword=biblioauthorprogram]%
        \printbibliography[heading=nobibheading, section=1, env=countauthor,             keyword=biblioauthor]%
        \printbibliography[heading=nobibheading, section=1, env=countauthorvakscopuswos, filter=vakscopuswos]%
        \printbibliography[heading=nobibheading, section=1, env=countauthorscopuswos,    filter=scopuswos]%
        %
        \nocite{*}%
        %
        {\publications} Основные результаты по теме диссертации изложены в~\arabic{citeauthor}~печатных изданиях,
        \arabic{citeauthorvak} из которых изданы в журналах, рекомендованных ВАК\sloppy%
        \ifnum \value{citeauthorscopuswos}>0%
            , \arabic{citeauthorscopuswos} "--- в~периодических научных журналах, индексируемых Web of~Science и Scopus\sloppy%
        \fi%
        \ifnum \value{citeauthorconf}>0%
            , \arabic{citeauthorconf} "--- в~тезисах докладов.
        \else%
            .
        \fi%
        \ifnum \value{citeregistered}=1%
            \ifnum \value{citeauthorpatent}=1%
                Зарегистрирован \arabic{citeauthorpatent} патент.
            \fi%
            \ifnum \value{citeauthorprogram}=1%
                Зарегистрирована \arabic{citeauthorprogram} программа для ЭВМ.
            \fi%
        \fi%
        \ifnum \value{citeregistered}>1%
            Зарегистрированы\ %
            \ifnum \value{citeauthorpatent}>0%
            \formbytotal{citeauthorpatent}{патент}{}{а}{}\sloppy%
            \ifnum \value{citeauthorprogram}=0 . \else \ и~\fi%
            \fi%
            \ifnum \value{citeauthorprogram}>0%
            \formbytotal{citeauthorprogram}{программ}{а}{ы}{} для ЭВМ.
            \fi%
        \fi%
        % К публикациям, в которых излагаются основные научные результаты диссертации на соискание учёной
        % степени, в рецензируемых изданиях приравниваются патенты на изобретения, патенты (свидетельства) на
        % полезную модель, патенты на промышленный образец, патенты на селекционные достижения, свидетельства
        % на программу для электронных вычислительных машин, базу данных, топологию интегральных микросхем,
        % зарегистрированные в установленном порядке.(в ред. Постановления Правительства РФ от 21.04.2016 N 335)
    \end{refsection}%
    \begin{refsection}[bl-author, bl-registered]
        % Это refsection=2.
        % Процитированные здесь работы:
        %  * попадают в авторскую библиографию, при usefootcite==0 и стиле `\insertbiblioauthorimportant`.
        %  * ни на что не влияют в противном случае
        \nocite{vakbib2}%vak
        \nocite{patbib1}%patent
        \nocite{progbib1}%program
        \nocite{bib1}%other
        \nocite{confbib1}%conf
    \end{refsection}%
        %
        % Всё, что вне этих двух refsection, это refsection=0,
        %  * для диссертации - это нормальные ссылки, попадающие в обычную библиографию
        %  * для автореферата:
        %     * при usefootcite==0, ссылка корректно сработает только для источника из `external.bib`. Для своих работ --- напечатает "[0]" (и даже Warning не вылезет).
        %     * при usefootcite==1, ссылка сработает нормально. В авторской библиографии будут только процитированные в refsection=0 работы.
}

При использовании пакета \verb!biblatex! будут подсчитаны все работы, добавленные
в файл \verb!biblio/author.bib!. Для правильного подсчёта работ в~различных
системах цитирования требуется использовать поля:
\begin{itemize}
        \item \texttt{authorvak} если публикация индексирована ВАК,
        \item \texttt{authorscopus} если публикация индексирована Scopus,
        \item \texttt{authorwos} если публикация индексирована Web of Science,
        \item \texttt{authorconf} для докладов конференций,
        \item \texttt{authorpatent} для патентов,
        \item \texttt{authorprogram} для зарегистрированных программ для ЭВМ,
        \item \texttt{authorother} для других публикаций.
\end{itemize}
Для подсчёта используются счётчики:
\begin{itemize}
        \item \texttt{citeauthorvak} для работ, индексируемых ВАК,
        \item \texttt{citeauthorscopus} для работ, индексируемых Scopus,
        \item \texttt{citeauthorwos} для работ, индексируемых Web of Science,
        \item \texttt{citeauthorvakscopuswos} для работ, индексируемых одной из трёх баз,
        \item \texttt{citeauthorscopuswos} для работ, индексируемых Scopus или Web of~Science,
        \item \texttt{citeauthorconf} для докладов на конференциях,
        \item \texttt{citeauthorother} для остальных работ,
        \item \texttt{citeauthorpatent} для патентов,
        \item \texttt{citeauthorprogram} для зарегистрированных программ для ЭВМ,
        \item \texttt{citeauthor} для суммарного количества работ.
\end{itemize}
% Счётчик \texttt{citeexternal} используется для подсчёта процитированных публикаций;
% \texttt{citeregistered} "--- для подсчёта суммарного количества патентов и программ для ЭВМ.

Для добавления в список публикаций автора работ, которые не были процитированы в
автореферате, требуется их~перечислить с использованием команды \verb!\nocite! в
\verb!Synopsis/content.tex!.
 % Характеристика работы по структуре во введении и в автореферате не отличается (ГОСТ Р 7.0.11, пункты 5.3.1 и 9.2.1), потому её загружаем из одного и того же внешнего файла, предварительно задав форму выделения некоторым параметрам

%Диссертационная работа была выполнена при поддержке грантов ...

%\underline{\textbf{Объем и структура работы.}} Диссертация состоит из~введения, четырех глав, заключения и~приложения. Полный объем диссертации \textbf{ХХХ}~страниц текста с~\textbf{ХХ}~рисунками и~5~таблицами. Список литературы содержит \textbf{ХХX}~наименование.

%\newpage
\section*{Содержание работы}

Во \underline{\textbf{введении}} обосновывается актуальность избранной темы, характеризуется степень ее разработанности,
определяются цели и задачи диссертационного исследования, осуществляется выбор предмета и объекта исследования,
раскрывается научная новизна, определяются методологические основания исследования,
теоретическая и практическая значимость полученных автором результатов,
формулируются положения, выносимые на защиту.

\underline{\textbf{Первая глава}} содержит математические основы кинетического описания разрежённого газа
на основе уравнения Больцмана, его линеаризации, а также широко используемых модельных уравнений.
Отдельное внимание уделено современной нелинейной асимптотической теории для слаборазрежённого газа,
где решение стационарного кинетического уравнения возможно представить в виде суммы
гидродинамического решения и поправки кинетического пограничного слоя (слоя Кнудсена).
Рассматриваются как медленные неизотермические течения с конечными числами Рейнольдса,
так и задачи с возникновением вязкого пограничного слоя (слоя Прандтля) при конечных числах Маха.
Обосновывается методика постановки третьей краевой задачи для решения
соответствующих уравнений гидродинамического типа.

\underline{\textbf{Вторая глава}} посвящена численным методам решения как уравнений гидродинамического типа,
так и непосредственно уравнения Больцмана.
Основное внимание уделяется построению консервативных разностных схем.
Для этой цели при решении систем дифференциальных уравнений в частных производных
используется метод конечных объёмов.
Для решения уравнений КГФ используется модификация алгоритма SIMPLE "--- стандартного метода решения
стационарных уравнений Навье"--~Стокса.
Излагается сравнительный обзор существующих методов решения уравнения Больцмана.
Избранный в работе подход к решению уравнения Больцмана основывается на работах Черемисина Ф.~Г.,
обобщённых на случай неравномерных сеток в пространстве скоростей.
Подробно излагается методика построения численной схемы, позволяющей достигнуть
второго порядка точности во всём семимерном фазовом пространстве:
по времени, по физическим и скоростным координатам.
Для этого используются симметричная схема расщепления, TVD-схемы для аппроксимации
бесстолкновительного уравнения Больцмана и проекционный метод дискретных скоростей
для вычисления интеграла столкновений.
Кроме стандартного двухточечного шаблона проецирования для равномерных сеток,
рассматриваются пятиточечная и семиточечная схемы.

\underline{\textbf{Третья глава}} состоит из нескольких классических задач молекулярной газовой динамики,
изучаемых с высокой точностью с помощью изложенных выше методов.
Сначала рассматривается плоское течение Куэтта для широкого диапазона чисел Кнудсена (0.01 --- 100)
и чисел Маха (0.1 -- 5). Анализируются полученные профили макроскопических величин в зависимости от
этих параметров. Верификация результатов проводится по известному численному решению линейной задачи Куэтта,
а также с помощью статистического моделирования (метод DSMC).
Кроме того, детально рассматриваются срезы функции распределения скоростей, её анизотропичность и разрывы на границе.
Задача Соне"--~Бобылева также детально изучается как в помощью асимптотического решения,
так и численного решения уравнения Больцмана. Демонстрируются ненавье"--~стоксовские эффекты,
а также сходимость кинетического решения именно к решению уравнений КГФ,
а не Навье"--~Стокса в континуальном пределе (эффект призрака).
Показывается, что решение третьей краевой задачи позволяет получить приближённое решение
для малых чисел Кнудсена, которое с хорошей точностью отражает прямое решение уравнения Больцмана.
Наконец, рассматриваются некоторые задачи термострессовой конвекции, а именно
течение газа между некоаксиальными цилиндрами и коаксиальными эллиптическими цилиндрами.
В континуальном пределе проводится параметрический анализ этих задач для различных соотношений температур,
расстоянию между осями некоаксиальных цилиндров, углу поворота между главными осями эллипсов.
Детально изучается силы, действующие на неподвижные равномерно нагретые тела со стороны газа,
имеющие второй порядок малости по числу Кнудсена.
Численное решение уравнения Больцмана свидетельствует о том, что уже для \(\Kn\gtrapprox0.01\)
тепловое скольжение существенно превалирует над термострессовой конвекцией.

В \underline{\textbf{заключении}} приведены основные результаты работы, которые заключаются в следующем:
%% Согласно ГОСТ Р 7.0.11-2011:
%% 5.3.3 В заключении диссертации излагают итоги выполненного исследования, рекомендации, перспективы дальнейшей разработки темы.
%% 9.2.3 В заключении автореферата диссертации излагают итоги данного исследования, рекомендации и перспективы дальнейшей разработки темы.
\begin{enumerate}

\item Все полученные результаты могут быть обобщены для произвольного потенциала
практически без увеличения вычислительных затрат.

\item в расширении области применимости асимтотического решения за счёт
особой методики постановки граничных условий;

\item в комплексной верификации проекционного метода дискретных скоростей
для широкого круга практических задач;

\item С точки зрения точности результатов и вычислительных затрат асимптотическое решение
является оптимальным выбором для моделирования течений в области \emph{малых чисел Кнудсена}.
Однако такое решение доступно только для узкого класса краевых задач с гладкой границей,
поэтому в большей степени оно должно служить для верификации численных методов общего назначения.
Проекционный метод на равномерной скоростной сетке "--- один из таких.
Полученные с его помощью результаты демонстрируют хорошую сходимость к асимптотическому решению,
однако требуют большого числа итераций для достижения стационарного состояния.
\item Для \emph{чисел Кнудсена близких к единице} проекционный метод позволяет получить
более точное решение по сравнению с методами прямого статистического моделирования,
для которых характерны значительные флуктуации функции распределения.
Особенно это касается малых чисел Маха, где статистические флуктуации не позволяют
получить адекватную картину течений.
\item Эффективность проекционного метода, как и любого метода дискретных скоростей,
на равномерной сетке в скоростном пространстве снижается для \emph{больших чисел Кнудсена}
ввиду значительных градиентов функции распределения во всей области, занимаемой газом.
Использование в этом случае неравномерной скоростной сетки даёт возможность
аппроксимировать функцию распределения с высокой точностью. Однако поскольку методы
дискретных скоростей подразумевают постоянство скоростной сетки во всём физическом
пространстве, то такой подход не может быть применён непосредственно к задачам
с произвольной геометрией и требует дальнейшего существенного развития.
\item Для \emph{медленных неизотермических течений} численное решение уравнения Больцмана
сходится к решению уравнений КГФ, получаемых в ходе асимптотического анализа.
При этом смешивание граничных условий разных порядков позволяет получить
температурное поле не только в континуальном пределе, но и для малых чисел Кнудсена.
В последнем случае оно является приближённым, но не асимптотическим решением
следующего порядка.

\item Проекционный метод, как и любой метод дискретных скоростей, сталкивается с проблемой
эффективной дискретизации. Области пространства скоростей,
где наблюдаются значительные перепады функции распределения, представляют известные трудности
для достижения высокой точности аппроксимации.

\item По сравнению с другими методами, проекционный метод на неравномерных сетках
позволяет достичь повышенной точности численного анализа плоского приграничного слоя
от линейных и вплоть до гиперзвуковых течений для широкого диапазона чисел Кнудсена.

\item Использование всех совместимых с уравнениями КГФ граничных условий первого и второго порядка
позволяет существенно улучшить точность асимптотического решения

\end{enumerate}


%\newpage

\ifdefmacro{\microtypesetup}{\microtypesetup{protrusion=false}}{} % не рекомендуется применять пакет микротипографики к автоматически генерируемому списку литературы
\ifnumequal{\value{bibliosel}}{0}{% Встроенная реализация с загрузкой файла через движок bibtex8
  \renewcommand{\bibname}{\large \authorbibtitle}
  \nocite{*}
  \insertbiblioauthor           % Подключаем Bib-базы
  %\insertbiblioother   % !!! bibtex не умеет работать с несколькими библиографиями !!!
}{% Реализация пакетом biblatex через движок biber
  \ifnumgreater{\value{usefootcite}}{0}{
  \insertbiblioauthorcited      % Вывод процитированных в автореферате работ автора
  }{
  \insertbiblioauthor           % Вывод всех работ автора
%  \insertbiblioauthorgrouped    % Вывод всех работ автора, сгруппированных по источникам
%  \insertbiblioauthorimportant  % Вывод наиболее значимых работ автора (определяется в файле characteristic во второй section)
  \insertbiblioother            % Вывод списка литературы, на которую ссылались в тексте автореферата
  }
}
\ifdefmacro{\microtypesetup}{\microtypesetup{protrusion=true}}{}

