%%% Микротипографика %%%
%\ifnumequal{\value{draft}}{0}{% Только если у нас режим чистовика
%    \usepackage[final]{microtype}[2016/05/14] % улучшает представление букв и слов в строках, может помочь при наличии отдельно висящих слов
%}{}

