\chapter*{Список сокращений и условных обозначений}             % Заголовок
\addcontentsline{toc}{chapter}{Список сокращений и условных обозначений}  % Добавляем его в оглавление
\noindent
\begin{tabularx}{\textwidth}{r X}
  \textbf{BEM} & Boundary element method, метод граничных элементов\\
  \textbf{CST MWS} & Computer Simulation Technology Microwave Studio
  программа для компьютерного моделирования уравнений Максвелла\\
  \textbf{DDA} & Discrete dipole approximation, приближение дискретиных диполей\\
  \textbf{FDFD} & Finite difference frequency domain, метод конечных
  разностей в частотной области\\
  \textbf{КЭН} & Кандидат экономических наук\\
  \textbf{РАН} & Российская академия наук\\
  \textbf{РИТЭГ} & Радиоизотопный термоэлектрический генератор\\
  \textbf{PDF} & Portable Document Format\\
\end{tabularx}
