\chapter{Численное решение уравнения Больцмана} \label{chapt:numerics}

\section{Обзор вычислительных методов} \label{sect:review}

%%% численные методы
Until 1949 the subject of kinetic theory was essentially
coextensive with the Chapman-Enskog theory. The ralson
d 'etre of the Boltzmann equation was to compute transport coefficients in order to complete the classical equations of continuum mechanics. The phrase "solve the Boltzmann equation" is synonymous, in this era, with the operation of solving the Hilbert Integral equation. The concept of the Boltzmann equation as an entity in Itself, to be solved subject to initial and boundary conditions was missing (except for Carleman, whose singular relation to the subject was only revealed posthumously [Carleman, 1957]).
The break with this preoccupation with transport coefficients was made by the introduction of the concept
of Interpolating different levels of description of a
gas [Grad, 1949] between the (at that time) almost inaccessible Boltzmann description via a distribution function and the macroscopic description via fluid moments through the Chapman-Enskog hierarchy. Polynomial and moment approximations gave an ad hoc scheme for this interpolation.


% модели -- когда сложно построить прямой численный метод
% какие бывают
% Стохастические методы -- на основе случайного процесса (стох модели), который в пределе даёт УБ
% лучше всего -- синтез стохастики и детерминизма
% модель дискретного газа


%%% Прямые методы решения УБ -- решаем диффуры, численное интегрирование
Среди численных методов решения уравнения Больцмана можно выделить два широких класса:
\begin{itemize}
    \item методы дискретных скоростей (ординат),
    \item спектральные методы.
\end{itemize}


Их более чем полувековое развитие шло по пути достижения трёх основных свойств:
\begin{itemize}
    \item сохранение массы, импульса и энергии (консервативность);
    \item выполнение H-теоремы;
    \item положительность функции распределения.
\end{itemize}
% Можно нарушать консервативность, но на очень подробных сетках. ~\cite{Aoki}
% про моментные и лэттис

\subsection{Стохастические методы}

%%% DSMC
Для численного анализа течений разрежённого газа (особенно стационарных) долгое время основным
считался метод \emph{прямого статистического моделирования} (DSMC).
Впервые предложенный Г.\,А.~Бёрдом в 1963 году~\cite{Bird1963} на основе общих физических соображений,
он получил широкое распространение в прикладных областях~\cite{Bird1994}.
Позже вычислительная эффективность метода была улучшена на основе более совершенного алгоритма
выбора сталкивающихся частиц~\cite{Bird1989}.
М.\,С.~Иванов и С.\,В.~Рогазинский предложили альтернативную численную схему мажорантной частоты~\cite{Ivanov1988}.

%%% Стохастические модели разреженного газа
\emph{Стохастические модели} разрежённого газа впервые рассмотрены М.\,А.~Леонтовичем
ещё в 1935 году~\cite{Leontovich1935}, но только в 1992 году В.~Вагнер показал сходимость
полумарковского метода Бёрда к уравнению Больцмана~\cite{Wagner1992},
основываясь на результатах А.\,В.~Скорохода~\cite{Skorokhod1983} и С.\,Н.~Смирнова~\cite{Smirnov1989}.
До этого времени предлагались альтернативные модели, получаемые эвристически из уравнения Больцмана.
В частности, В.\,Е.~Яницкий и О.\,М.~Белоцерковский разработали стохастический метод~\cite{Yanitskij1975}
на основе строго марковского процесса эволюции модели Каца"--~Леонтовича~\cite{Kac1965, Kac1973},
которая асимптотически эквивалента уравнению Больцмана, однако строгое доказательство~\cite{Babovsky1989}
сходимости к уравнению Больцмана было получено только для метода Нанбу"--~Бабовского~\cite{Nanbu1980, Babovsky1986}.

%%% Недостатки DSMC и пути решения
К основным недостаткам метода DSMC относятся высокий уровень статистического шума
и значительный рост вычислительных затрат при стремлении к континуальному пределу.
%%% а) линейные течения
В последние годы предлагаются различные методы уменьшения дисперсии (variance reduction) для слабо возмущённых течений
(information preservation~\cite{Fan2001, Sun2002}, time-relaxed~\cite{Pareschi2001, Pareschi2011},
low-variance~\cite{Hadji2005, Hadji2007, Hadji2011}).
Основой этих \emph{гибридных} подходов является априорное представление о функции распределения
и применение статистического моделирования только к возмущённой части решения (алгебраическая декомпозиция).
Обобщение девиационного стохастического подхода, допускающего отрицательные частицы,
на нелинейное уравнение Больцмана приводит к значительным вычислительным трудностям~\cite{Wagner2008}.
Активно развиваются гибридные методы, использующие классическую схему Бёрда
(moment-guide~\cite{Dimarco2011, Dimarco2013}, convex combination~\cite{Dimarco2008, Caflisch2016}).
%%% б) гиперзвуковые течения
Особую сложность представляют также течения с высоким перепадом плотности (например, гиперзвуковые),
поскольку очень малая часть ансамбля модельных частиц попадает в области наиболее разреженного газа (например, донную).
С кинетической точки зрения, это многомасштабные задачи с широким диапазоном чисел Кнудсена.
Стохастический метод взвешенных частиц позволяет адаптировать уровень дисперсии посредством
их деления и аннигиляции, однако ценой высокой сложности алгоритма~\cite{Rjasanow1996, Rjasanow2005}.
Несмотря на то что в широком круге задач метод DSMC позволяет достичь инженерной точности,
множество тонких эффектов остаются за гранью его реальной разрешающей способности.

~\cite{Villani1999divergent} Kinetic Force Method and rotating quasi-particles~\cite{Saveliev2002}

\subsection{Методы дискретных скоростей}

%%% Методы дискретных скоростей
\emph{Метод дискретных скоростей} восходит к работам 40-х годов Нобелевского лауреата С.~Чандрасекара
в области теории излучения~\cite{Chandrasekhar1950}, ещё до появления методов прямого статистического моделирования.
Первые значительные успехи, связанные с применением метода ДС к численному решению уравнения Больцмана,
были получены в США Дж.~Хэвилендом~\cite{Haviland1965, Haviland1970}, а также А.~Нордсиком и Б.~Хиксом~\cite{Nordsieck1966, Nordsieck1970}.
Они использовали метод Монте-Карло для вычисления пятимерного интеграла столкновений
и простейшую коррекцию функции распределения для получения устойчивой численной схемы.
Начиная с 1965 года, после ввода в эксплуатацию БЭСМ-6, \emph{метод Хикса"--~Йена"--~Нордсика}~\cite{Nordsieck1966, Yen1984}
активно развивался в Вычислительном центре АН СССР (Ф.\,Г.~Черемисин, В.\,В.~Аристов и др.).
На первых ЭВМ основную трудность представляли ограничения в объёме памяти~\cite{Tcheremissine1970}.
Операторное расщепление уравнения Больцмана и полиномиальная коррекция существенно повысили точность~\cite{Tcheremissine1980}.
Предложенный консервативный метод успешно применялся во многих прикладных областях.

\subsubsection{Упрощение столкновительного оператора}

%%% Модельные уравнения
Ввиду высокой сложности столкновительного члена в уравнении Больцмана, начиная с середины XX века,
активно применяются упрощённые так называемые \emph{модельные уравнения}.
В 1954 году группа М.~Крука~\cite{Krook1954} и независимо П.~Веландер~\cite{Welander1954}
предложили простейшую кинетичекую модель с фиксированным числом Прандтля \(\Pr=1\).
Этого недостатка лишены \emph{эллипсоидальная статистическая} (ЭС) модель Л.~Холвея~\cite{Holway1963, Holway1966}
и \emph{S-модель} Е.\,М.~Шахова~\cite{Shakhov1968}.
S-модель, вообще говоря, применима только для слабо неравновесного газа,
в противном случае функция распределения может принимать отрицательные значения для высоких молекулярных скоростей.
Справедливость H-теоремы для ЭС-модели была доказана лишь в 1999 году~\cite{Perthame2000}.
Простота модельного столкновительного оператора обуславливает широкое разнообразие соответствующих численных методов,
бурно развивающихся по настоящее время~\cite{Dimarco2014}.
Практические расчёты показывают, что разрежённый газ с невысокой степенью неравновесности в большинстве случаев
корректно описывается модельными уравнениями, однако не представляется возможным произвести какие-либо
априорные оценки отклонения от истинного решения уравнения Больцмана.

\subsubsection{Дискретные модели}

%%% Модели дискретных скоростей
Одновременно с численными методами глубокое развитие получили математические \emph{модели дискретных скоростей}.
Простейшая дискретная модель, содержащая только 2 возможных вектора скорости,
была рассмотрена ещё в фундаметальном труде Т.~Карлемана~\cite{Carleman1957}.
Дж.~Бродуэлл в 1964 году использовал 6 и 8 скоростей для численного анализа простейших задач
разреженного газа~\cite{Broadwell1964shock, Broadwell1964shear}.
Его успех вызвал интерес у французских математиков Р.~Гатиньоль~\cite{Gatignol1975} и А.~Кабанна~\cite{Cabannes1980},
впервые представившие систематическую теорию газа дискретных скоростей.
С.\,К.~Годунов и У.\,М.~Султангазин были в числе пионеров построения формальной теории~\cite{Sultangazin1971}.
% А.\,В.~Бобылевым и К.~Черчиньяни~\cite{Bobylev1999dvm}.
На рубеже 80-х и 90-х годов были получены первые численные решения на основе моделей
с большим количеством дискретных скоростей~\cite{Goldstein1989, Inamuro1990}.
В отличие от метода Хикса"--~Йена"--~Нордсика, при вычислении интеграла столкновений используются
только те четвёрки скоростей, которые сохраняют импульс и кинетическую энергию.
\emph{Консервативность на микроскопическом уровне} позволяла достичь высокой точности,
однако в первоначальной постановке метод имел квадратичную сложность.
К.~Бюе повысил эффективность метода с помощью некоторых стохастических приёмов~\cite{Buet1996},
однако вопрос сходимости дискретных моделей к уравнению Больцмана долгое время оставался открытым.
Положительный ответ удалось найти только на основе современных достижений теории чисел~\cite{Palczewski1997}.
Кроме того, было показано, что конечно-разностная схема для дискретных скоростей
аппроксимирует слабые решения Ди-Перна"--~Лионса~\cite{Mischler1997, Palczewski1998}.
% представление Карлемана равномерное распределени проще доказать В.~Панфёровым и А.~Гейнцем~\cite{Panferov2002}

\subsubsection{Современные подходы}

%%% Fainsilber
%The models studied here are constructed by discretizing, one at a time, the it- erated integrals (2). An alternative way of writing this integral was introduced by Carleman [5]. Using that v′ − v and v∗′ − v are orthogonal, one can write (here we specialize to d = 3)
%Q(f,f)(v)= (f(v′)f(v∗′)−f(v)f(v∗))q(w,cosθ) ′ 2 dE(v∗′)dv′, R3 Ev,v′ |v − v |
%where Ev,v′ is the plane that contains v and is orthogonal to v′ −v, and where dE(v∗′ ) is the Euclidean measure on this plane. Panferov and Heintz [17] have analyzed a DVM based on this iterated integral, and proved that the method is consistent with the continuous model. This is somehow easier, because on a given plane, one can find all integer points by solving @@@linear Diophantine equations@@@. However, the density of points depends strongly on v′ − v, and so it is far from trivial to prove the consistency.


%%% Современные методы дискретных скоростей
Оценка скорости сходимости, полученная на основе гипотезы Рамануджана~\cite{Ramanujan1916},
говорит о низкой вычислительной эффективности численного анализа уравнения Больцмана
на основе классических моделей дискретных скоростей. Основная трудность связана с малым количеством
пар разлётных скоростей для выбранной столкновительной пары.
К.~Бюе, С.~Кордье и П.~Дегон предложили несколько регуляризационных подходов,
сохраняющих консервативность численной схемы в слабой форме~\cite{Buet1998}.
% smoothing of the collision spheres
Один из них основан на методе пространственной делокализации молекулярного взаимодействия,
разработанном в упомянутой ранее работе А.\,Я.~Повзнера~\cite{Povzner1962} + Моргенштерн.
% smoothing of the mass molecules
Второй, напротив, при локальном сохранении импульса и энергии, допускает нарушения инвариатности массы.
Однако консервативное в слабом смысле размазывание (mollification) процесса столкновений
создаёт зависимость дискретного столкновительного оператора от функции распределения
или отдельных её моментов. Такое требование существенно ограничивает эффективность численной реализации.
% smoothing the collision integral ~\cite{Gorsch2002}
В 1997 году Ф.\,Г.~Черемисин предложил новый класс методов дискретных скоростей,
основанных на \emph{консервативном проецировании} разлётных частиц~\cite{Tcheremissine1997, Tcheremissine1998}.
Допуская использование почти любых столкновений,
проекционный метод в то же время лишён основного недостатка регуляризационных схем.
Специальная процедура интерполяции функции распределения обеспечивает точное выполнение H-теоремы~\cite{Tcheremissine2000}.
\emph{Проекционно"=интерполяционный метод} нашёл широкое применение,
однако сходимость метода и порядок аппроксимации не были строго исследованы до настоящего времени.

\subsection{Спектральные методы}
0. базис из дельта-функций --> DVM
1. моментные методы. Мотт-Смита, Грэда
2. фурье-базис Fourier discretization закончить про консервативную коррекцию
3. разрывный Галёркин кусочно-полиномиальная аппроксимация Limar1973
4. вейвлеты Лемоу

% основная цель -- убрать целочисленное интегрирование по углу
% нет консервативности и положительности, поэтому требуется коррекция
% The lack of discrete conservations in the spectral scheme (mass is preserved, whereas momentum and energy are approximated with spectral accuracy) is compensated by its higher accuracy and efficiency.



% Heintz2008: design a Fourier based deterministic method... Using Fourier transform for computation of Q(f,f) is attractive because one goes around the integration over the sphere in (2) and makes use of the convolution structure in Q(f,f). But Fourier based spectral approximations for discontinuous solutions lose accuracy because of the Gibbs phenomenon... The main idea here is to combine an adaptive approximation for discontinuous solutions f to the Boltzmann equation with Fourier spectral approximation for the smooth terms in the equation: the gain term Q^+(f,f) and the collision frequency q^-(f). It is based on the classical result that in the case of Maxwell molecules and hard potentials with cut off the gain term  Q^+(f,f) in the collision operator has certain smoothing properties and is actually smooth even for a discontinuous f. This property was found first in [Lions94p1,2] and later investigated in details in [Wennberg97Radon, Mouhot&Villani04]. The collision frequency q ðfÞ is also a smooth function because it is a convolution of f with a regular function.

Ю.\,Н.~Григорьеву и А.\,Н.~Михалицыну~\cite{Grigoriev1983},
для численного анализа процессов релаксации высоких моментов фукнции распределения


% Mouhot, Pareschi2006 make the decoupling assumption on the kernel => faster
% Pareschi, Russo2000 -- introduces some “filters” on the Fourier modes in order to restore the positivity-preservation of the scheme, which breaks the spectral accuracy.
% Filbet, Mouhot2011 -- first stability result for the spectral methods applied to the Boltzmann collision operator: a smooth balanced perturbation of the original equa- tion, in the sense of a perturbation by some small and mass-preservation (although not positivity-preserving) error term;
% Fonn2014 -- частных сумм по гиперболическим крестам рядов Фурье периодических функций многих переменных
% we expect that the solutions become radially symmetric much more quickly than they approach a Maxwellian
% In other words, for near-Maxwellian solutions, one should expect the perturbations from equilibrium to be radially symmetric after a short time.
Наконец, метод дискретных скоростей может быть интерпретирован как проекционный метод в пространстве дельта-функций.
Кусочно-полиномиальная аппроксимация функции распределения впервые была применена ещё Е.\,Ф.~Лимаром~\cite{Limar1973}.
А.~Майорана показал, как может быть построен консервативный разрывный метод Галёркина~\cite{Majorana2011}.

% Ender1988 -- внутренне консервативный численный метод


%%% Спектральные методы
Спектральные методы берут своё начало с работы 1996 года Л.~Парески и Б.~Пертама~\cite{Pareschi1996},
основанной на аналитических исследованиях А.\,В.~Бобылева~\cite{Bobylev1984}.
Однако в своей первоначальной постановке предложенные численные методы вычисления
больцмановского интеграла столкновений не удовлетворяют основным его свойствам:
\begin{itemize}
    \item сохранение массы, импульса и энергии;
    \item положительность функции распределения;
    \item равенство нулю для максвелловского распределения.
\end{itemize}
В 1994 появились первые консервативные методики на основе специального выбора
дискретных скоростей (Ф.~Рожье, Ж.~Шнайдер, Ч.~Тан, Ф.\,Л.~Варгезе),
однако для них характерна медленная сходимость решения.
В 1997 году Ф.\,Г.~Черемисин разработал консервативный проекционный метод дискретных скоростей,
позже для выполнения консервативных свойств в спектральных методах были предложены методы
условной оптимизации (А.\,В.~Бобылев, С.\,В.~Рязанов, И.\,М.~Гамба, С.\,Х.~Таркабхушанам).

% 1) экспоненциальная сходимость 2) консервативная коррекция 3) нарушение позитивности и феномен Гиббса 4)
% стабильность метода Фурье--Галёркина
% Mouhot,Pareschi2003 -- implementation, multiple resolution, link between grids
% The same problem for DG.

% нет смысла вычислять столкновительный интеграл точно на дискретной сетке, потому что аппроксимация на ней и так даёт большую ошибку
% поэтому метод Коробова и метод Mouhot, Pareschi, Rey13

1. Аналитическое исследование уравнения Больцмана сложно и трудоёмко,
точное решение возможно получить лишь для очень узкого спектра задач.

2. Некоторые классические задачи динамики разреженного газа удаётся выразить в квадратурах
в рамках линеаризованного уравнения Больцмана.

3. Зачастую некоторые качественные оценки поведения разреженного газа можно получить на основе модельных уравнений [],
однако в этом случае практически невозможно оценить получаемую ошибку, обусловленную приближённостью соответствующей модели.


\subsection{Неравномерные сетки}
1. Heintz13 -- where the velocity space is non-uniformly discretized but the frequency space is equally divided.
2. Morris, Varghese

% из DVM лучше метод Черема обощается (более локальный)

\subsection{Многомасштабные методы}

% когда несколько масштабов?
% coupling (domain-decomposition), asymptotic-preserving

%AP: 1) IMEX 2) gas-kinetic Xu

\section{Проекционно-интерполяционный метод дискретных скоростей} \label{sect:method}

\subsection{Операторное расщепление}
\subsection{Метод конечных объёмов}
\subsection{Метод дискретных скоростей}
\subsection{Проекционно-интерполяционный подход}
\subsection{Решение задачи Коши}
\subsection{Сохранение положительности}
\subsection{Шаблоны проецирования}

%\newpage
%============================================================================================================================

\clearpage
