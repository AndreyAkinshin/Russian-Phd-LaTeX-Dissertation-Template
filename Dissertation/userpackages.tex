\usepackage{tabu, tabulary}  %таблицы с автоматически подбирающейся шириной столбцов
\usepackage{fr-longtable}    %ради \endlasthead

% Листинги с исходным кодом программ
\usepackage{fancyvrb}
\usepackage{listings}

% Русская традиция начертания греческих букв
%\usepackage{upgreek} % прямые греческие ради русской традиции

% Микротипографика
%\ifnumequal{\value{draft}}{0}{% Только если у нас режим чистовика
%    \usepackage[final]{microtype}[2016/05/14] % улучшает представление букв и слов в строках, может помочь при наличии отдельно висящих слов
%}{}

% Отметка о версии черновика на каждой странице
% Чтобы работало надо в своей локальной копии по инструкции
% https://www.ctan.org/pkg/gitinfo2 создать небходимые файлы в папке
% ./git/hooks
% If you’re familiar with tweaking git, you can probably work it out for
% yourself. If not, I suggest you follow these steps:
% 1. First, you need a git repository and working tree. For this example,
% let’s suppose that the root of the working tree is in ~/compsci
% 2. Copy the file post-xxx-sample.txt (which is in the same folder of
% your TEX distribution as this pdf) into the git hooks directory in your
% working copy. In our example case, you should end up with a file called
% ~/compsci/.git/hooks/post-checkout
% 3. If you’re using a unix-like system, don’t forget to make the file executable.
% Just how you do this is outside the scope of this manual, but one
% possible way is with commands such as this:
% chmod g+x post-checkout.
% 4. Test your setup with “git checkout master” (or another suitable branch
% name). This should generate copies of gitHeadInfo.gin in the directories
% you intended.
% 5. Now make two more copies of this file in the same directory (hooks),
% calling them post-commit and post-merge, and you’re done. As before,
% users of unix-like systems should ensure these files are marked as
% executable.
\ifnumequal{\value{draft}}{1}{% Черновик
   \IfFileExists{.git/gitHeadInfo.gin}{                                        
      \usepackage[mark,pcount]{gitinfo2}
      \renewcommand{\gitMark}{rev.\gitAbbrevHash\quad\gitCommitterEmail\quad\gitAuthorIsoDate}
      \renewcommand{\gitMarkFormat}{\color{Gray}\small\bfseries}
   }{}
}{}

%для того, чтобы считать из диссертации счеткики (кол-во страниц, ссылок и т.п.) и записать значения в dissertation.aux и позже в автореферат, решение подробно расписано на страничке:
%http://tex.stackexchange.com/q/337263/44348
\usepackage{zref-abspage, zref-lastpage,zref-user}

\newcounter{chapterc} %for Counting chapters in dissertation % нужно заменить на работающий счетчик глав, тогда разметку \label{chapterfc} по тексту можно убрать
\newcounter{appendixc} %for Counting appendices in dissertation % нужно заменить на работающий счетчик приложений, тогда разметку \label{appendixc} по тексту можно убрать


%формируем значения для экспорта в synopsis
\makeatletter
\zref@newprop{mchapter}[0]{\thechapterc} %отдельный счетчик на главы, размеченный по тексту
\zref@newprop{mappendix}[0]{\theappendixc} %отдельный счетчик на главы, размеченный по тексту
\zref@newprop{mfigure}[0]{\totalfigures} %счетчик на рисунки, использует totalcount
\zref@newprop{mtable}[0]{\totaltables}  %счетчик на таблицы, использует totalcount
\zref@newprop{mrefs}[0]{\the\totvalue{citenum}} %счетчик на список литературы, использует totcount 
\zref@addprops{LastPage}{mchapter,mappendix,mfigure,mtable,mrefs} %добавляем все счетчики в zref и в dissertation.aux
\makeatother