\usepackage{fr-longtable}    %ради \endlasthead
\usepackage{relsize}

% Листинги с исходным кодом программ
\usepackage{fancyvrb}
\usepackage{listings}
\lccode`\~=0\relax %Без этого хака из-за особенностей пакета listings перестают работать конструкции с \MakeLowercase и т. п. в (xe|lua)latex

% Русская традиция начертания греческих букв
\usepackage{upgreek} % прямые греческие ради русской традиции

%%% Микротипографика
%\ifnumequal{\value{draft}}{0}{% Только если у нас режим чистовика
%    \usepackage[final, babel, shrink=45]{microtype}[2016/05/14] % улучшает представление букв и слов в строках, может помочь при наличии отдельно висящих слов
%}{}

% Отметка о версии черновика на каждой странице
% Чтобы работало надо в своей локальной копии по инструкции
% https://www.ctan.org/pkg/gitinfo2 создать небходимые файлы в папке
% ./git/hooks
% If you’re familiar with tweaking git, you can probably work it out for
% yourself. If not, I suggest you follow these steps:
% 1. First, you need a git repository and working tree. For this example,
% let’s suppose that the root of the working tree is in ~/compsci
% 2. Copy the file post-xxx-sample.txt (which is in the same folder of
% your TEX distribution as this pdf) into the git hooks directory in your
% working copy. In our example case, you should end up with a file called
% ~/compsci/.git/hooks/post-checkout
% 3. If you’re using a unix-like system, don’t forget to make the file executable.
% Just how you do this is outside the scope of this manual, but one
% possible way is with commands such as this:
% chmod g+x post-checkout.
% 4. Test your setup with “git checkout master” (or another suitable branch
% name). This should generate copies of gitHeadInfo.gin in the directories
% you intended.
% 5. Now make two more copies of this file in the same directory (hooks),
% calling them post-commit and post-merge, and you’re done. As before,
% users of unix-like systems should ensure these files are marked as
% executable.
\ifnumequal{\value{draft}}{1}{% Черновик
   \IfFileExists{.git/gitHeadInfo.gin}{
      \usepackage[mark,pcount]{gitinfo2}
      \renewcommand{\gitMark}{rev.\gitAbbrevHash\quad\gitCommitterEmail\quad\gitAuthorIsoDate}
      \renewcommand{\gitMarkFormat}{\rmfamily\color{Gray}\small\bfseries}
   }{}
}{}

\newcommand{\astfootnote}[1]{
	\let\oldthefootnote=\thefootnote
	\setcounter{footnote}{0}
	\renewcommand{\thefootnote}{\fnsymbol{footnote}}
	\footnote{#1}
	\let\thefootnote=\oldthefootnote
}

\newcommand\extrafootertext[1]{
    \bgroup
    \renewcommand\thefootnote{\fnsymbol{footnote}}
    \renewcommand\thempfootnote{\fnsymbol{mpfootnote}}
    \footnotetext[0]{#1}
    \egroup
}
