%%% Поля и разметка страницы %%%
\usepackage{lscape} % Для включения альбомных страниц

%%% Кодировки и шрифты %%%
\usepackage[utf8]{inputenc} % Кодировка utf8
\usepackage[english, russian]{babel} % Языки: русский, английский
\usepackage{pscyr} % Нормальные шрифты

%%% Математические пакеты %%%
\usepackage{amsthm,amsfonts,amsmath,amssymb,amscd,mathtext}

%%% Оформление абзацев %%%
\usepackage{indentfirst} % Красная строка

%%% Цвета %%%
\usepackage[usenames]{color}
\usepackage{color}
\usepackage{colortbl}

%%% Таблицы %%%
\usepackage{longtable}
\usepackage[singlelinecheck=off]{caption}

%%% Список литературы %%%
\usepackage{cite}
\makeatletter
\renewcommand{\@biblabel}[1]{#1.} % Заменяем библиографию с квадратных скобок на точку:
\makeatother

%%% Гиперссылки %%%
%\usepackage[plainpages=false,pdfpagelabels=false]{hyperref}

%%% Изображения %%%
\usepackage{graphicx} % Подключаем пакет работы с графикой
\graphicspath{{images/}} % Пути к изображениям


%%% Прочее %%%
\usepackage{array}
\usepackage{fontenc}
\usepackage{textcomp}